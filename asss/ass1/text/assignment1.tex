
\documentclass{article}
%%%%%%%%%%%%%%%%%%%%%%%%%%%%%%%%%%%%%%%%%%%%%%%%%%%%%%%%%%%%%%%%%%%%%%%%%%%%%%%%%%%%%%%%%%%%%%%%%%%%%%%%%%%%%%%%%%%%%%%%%%%%%%%%%%%%%%%%%%%%%%%%%%%%%%%%%%%%%%%%%%%%%%%%%%%%%%%%%%%%%%%%%%%%%%%%%%%%%%%%%%%%%%%%%%%%%%%%%%%%%%%%%%%%%%%%%%%%%%%%%%%%%%%%%%%%
\usepackage{amsmath}
\usepackage{amsfonts}
\usepackage{amssymb}
\usepackage{color}
\usepackage{url}
\usepackage{tikz}
\usepackage[nomarkers,notablist,nofiglist]{endfloat}
\usepackage{multirow, array}
\usepackage{graphicx}
\usepackage{subcaption}
\usepackage{setspace}
\usepackage[margin=1in]{geometry}

\setcounter{MaxMatrixCols}{10}
%TCIDATA{OutputFilter=LATEX.DLL}
%TCIDATA{Version=5.50.0.2960}
%TCIDATA{<META NAME="SaveForMode" CONTENT="1">}
%TCIDATA{BibliographyScheme=Manual}
%TCIDATA{Created=Thursday, June 03, 2021 22:10:23}
%TCIDATA{LastRevised=Sunday, June 20, 2021 12:47:32}
%TCIDATA{<META NAME="GraphicsSave" CONTENT="32">}
%TCIDATA{<META NAME="DocumentShell" CONTENT="Standard LaTeX\Blank - Standard LaTeX Article">}
%TCIDATA{CSTFile=40 LaTeX article.cst}

\doublespacing
\usetikzlibrary{shapes}
\usetikzlibrary{arrows.meta}
\tikzset{
diagonal fill/.style 2 args={fill={rgb,255:red,118; green,219; blue,128}, path picture={
\fill[#1, sharp corners] (path picture bounding box.south east) -|
(path picture bounding box.center) -- cycle;}},
reversed diagonal fill/.style 2 args={fill=yellow, path picture={
\fill[#1, sharp corners] (path picture bounding box.center) |- 
(path picture bounding box.south east) -- cycle;}}
}
\newtheorem{theorem}{Theorem}
\newtheorem{acknowledgement}[theorem]{Acknowledgement}
\newtheorem{algorithm}[theorem]{Algorithm}
\newtheorem{axiom}[theorem]{Axiom}
\newtheorem{case}[theorem]{Case}
\newtheorem{claim}[theorem]{Claim}
\newtheorem{conclusion}[theorem]{Conclusion}
\newtheorem{condition}[theorem]{Condition}
\newtheorem{conjecture}[theorem]{Conjecture}
\newtheorem{corollary}[theorem]{Corollary}
\newtheorem{criterion}[theorem]{Criterion}
\newtheorem{definition}[theorem]{Definition}
\newtheorem{example}[theorem]{Example}
\newtheorem{exercise}[theorem]{Exercise}
\newtheorem{lemma}[theorem]{Lemma}
\newtheorem{notation}[theorem]{Notation}
\newtheorem{problem}[theorem]{Problem}
\newtheorem{proposition}[theorem]{Proposition}
\newtheorem{remark}[theorem]{Remark}
\newtheorem{solution}[theorem]{Solution}
\newtheorem{summary}[theorem]{Summary}
\newenvironment{proof}[1][Proof]{\noindent\textbf{#1.} }{\ \rule{0.5em}{0.5em}}


\begin{document}

\title{ \textsc{COMM8102 Econometric Analysis} \\
\textsc{Assignment 1}}
\author{\textsc{Yuhao Liu} \\
%EndAName
{\small z5097536}}
\date{{\small \today}}
\maketitle

\section*{Excercise 2.4}

$\bigskip $%
\begin{eqnarray*}
\mathbb{E}\left[ \left. Y\right\vert X=0\right] &=&1\cdot \mathbb{P}\left(
\left. Y=1\right\vert X=0\right) +0\cdot \mathbb{P}\left( \left.
Y=0\right\vert X=0\right) \\
&=&1\cdot \frac{\mathbb{P}\left( Y=1,X=0\right) }{\mathbb{P}\left(
X=0\right) }+0\cdot \frac{\mathbb{P}\left( Y=0,X=0\right) }{\mathbb{P}\left(
X=0\right) } \\
&=&1\cdot \frac{0.4}{0.4+0.1}+0\cdot \frac{0.1}{0.4+0.1} \\
&=&0.8
\end{eqnarray*}

\begin{eqnarray*}
\mathbb{E}\left[ \left. Y\right\vert X=1\right] &=&1\cdot \mathbb{P}\left(
\left. Y=1\right\vert X=1\right) +0\cdot \mathbb{P}\left( \left.
Y=0\right\vert X=1\right) \\
&=&1\cdot \frac{\mathbb{P}\left( Y=1,X=1\right) }{\mathbb{P}\left(
X=1\right) }+0\cdot \frac{\mathbb{P}\left( Y=0,X=1\right) }{\mathbb{P}\left(
X=1\right) } \\
&=&1\cdot \frac{0.3}{0.3+0.2}+0\cdot \frac{0.2}{0.3+0.21} \\
&=&0.6
\end{eqnarray*}

\begin{eqnarray*}
\mathbb{E}\left[ \left. Y^{2}\right\vert X=0\right] &=&1^{2}\cdot \mathbb{P}%
\left( \left. Y=1\right\vert X=0\right) +0^{2}\cdot \mathbb{P}\left( \left.
Y=0\right\vert X=0\right) \\
&=&1\cdot \frac{\mathbb{P}\left( Y=1,X=0\right) }{\mathbb{P}\left(
X=0\right) }+0\cdot \frac{\mathbb{P}\left( Y=0,X=0\right) }{\mathbb{P}\left(
X=0\right) } \\
&=&1\cdot \frac{0.4}{0.4+0.1}+0\cdot \frac{0.1}{0.4+0.1} \\
&=&0.8
\end{eqnarray*}

\begin{eqnarray*}
\mathbb{E}\left[ \left. Y^{2}\right\vert X=1\right] &=&1^{2}\cdot \mathbb{P}%
\left( \left. Y=1\right\vert X=1\right) +0^{2}\cdot \mathbb{P}\left( \left.
Y=0\right\vert X=1\right) , \\
&=&1\cdot \frac{\mathbb{P}\left( Y=1,X=1\right) }{\mathbb{P}\left(
X=1\right) }+0\cdot \frac{\mathbb{P}\left( Y=0,X=1\right) }{\mathbb{P}\left(
X=1\right) }, \\
&=&1\cdot \frac{0.3}{0.3+0.2}+0\cdot \frac{0.2}{0.3+0.21} \\
&=&0.6
\end{eqnarray*}

\begin{eqnarray*}
Var\left[ \left. Y\right\vert X=0\right] &=&\mathbb{E}\left[ \left.
Y^{2}\right\vert X=0\right] -\left( \mathbb{E}\left[ \left. Y\right\vert X=0%
\right] \right) ^{2}, \\
&=&0.8-0.8^{2} \\
&=&0.16
\end{eqnarray*}

\begin{eqnarray*}
Var\left[ \left. Y\right\vert X=1\right] &=&\mathbb{E}\left[ \left.
Y^{2}\right\vert X=1\right] -\left( \mathbb{E}\left[ \left. Y\right\vert X=1%
\right] \right) ^{2}, \\
&=&0.6-0.6^{2} \\
&=&0.24
\end{eqnarray*}

\section*{Excercise 2.7}

\begin{eqnarray*}
\sigma \left( X\right) &=&Var\left[ \left. Y\right\vert X\right] \\
&=&\mathbb{E}\left[ \left. \left( Y-\mathbb{E}\left[ \left. Y\right\vert X%
\right] \right) ^{2}\right\vert X\right] \\
&=&\mathbb{E}\left[ \left. Y^{2}+\left( \mathbb{E}\left[ \left. Y\right\vert
X\right] \right) ^{2}-2\cdot Y\cdot \mathbb{E}\left[ \left. Y\right\vert X%
\right] \right\vert X\right] \\
&=&\mathbb{E}\left[ \left. Y^{2}\right\vert X\right] +\mathbb{E}\left[
\left. \left( \mathbb{E}\left[ \left. Y\right\vert X\right] \right)
^{2}\right\vert X\right] -2\cdot \mathbb{E}\left[ \left. Y\cdot \mathbb{E}%
\left[ \left. Y\right\vert X\right] \right\vert X\right] \\
&=&\mathbb{E}\left[ \left. Y^{2}\right\vert X\right] +\left( \mathbb{E}\left[
\left. Y\right\vert X\right] \right) ^{2}-2\cdot \mathbb{E}\left[ \left.
Y\right\vert X\right] \cdot \mathbb{E}\left[ \left. Y\right\vert X\right] \\
&=&\mathbb{E}\left[ \left. Y^{2}\right\vert X\right] -\left( \mathbb{E}\left[
\left. Y\right\vert X\right] \right) ^{2}
\end{eqnarray*}

\section{Excercise 2.18}

\subsection*{(a)}

\begin{eqnarray*}
\mathbf{Q}_{XX} &=&\mathbb{E}\left[ XX^{\intercal }\right] \\
&=&\left( 
\begin{array}{ccc}
\mathbb{E}\left[ 1\right] & \mathbb{E}\left[ X_{2}\right] & \mathbb{E}\left[
X_{3}\right] \\ 
\mathbb{E}\left[ X_{2}\right] & \mathbb{E}\left[ X_{2}^{2}\right] & \mathbb{E%
}\left[ X_{2}X_{3}\right] \\ 
\mathbb{E}\left[ X_{3}\right] & \mathbb{E}\left[ X_{2}X_{3}\right] & \mathbb{%
E}\left[ X_{3}^{2}\right]%
\end{array}%
\right) \\
&=&\left( 
\begin{array}{ccc}
1 & \mathbb{E}\left[ X_{2}\right] & \mathbb{E}\left[ \alpha _{1}+\alpha
_{2}X_{2}\right] \\ 
\mathbb{E}\left[ X_{2}\right] & \mathbb{E}\left[ X_{2}^{2}\right] & \mathbb{E%
}\left[ \alpha _{1}X_{2}+\alpha _{2}X_{2}^{2}\right] \\ 
\mathbb{E}\left[ \alpha _{1}+\alpha _{2}X_{2}\right] & \mathbb{E}\left[
\alpha _{1}X_{2}+\alpha _{2}X_{2}^{2}\right] & \mathbb{E}\left[ \alpha
_{1}^{2}+\alpha _{2}^{2}X_{2}^{2}+2\alpha _{1}\alpha _{2}X_{2}\right]%
\end{array}%
\right) \\
&=&\left( 
\begin{array}{ccc}
1 & \mathbb{E}\left[ X_{2}\right] & \alpha _{1}+\alpha _{2}\mathbb{E}\left[
X_{2}\right] \\ 
\mathbb{E}\left[ X_{2}\right] & \mathbb{E}\left[ X_{2}^{2}\right] & \alpha
_{1}\mathbb{E}\left[ X_{2}\right] +\alpha _{2}\mathbb{E}\left[ X_{2}^{2}%
\right] \\ 
\alpha _{1}+\alpha _{2}\mathbb{E}\left[ X_{2}\right] & \alpha _{1}\mathbb{E}%
\left[ X_{2}\right] +\alpha _{2}\mathbb{E}\left[ X_{2}^{2}\right] & \alpha
_{1}^{2}+2\alpha _{1}\alpha _{2}\mathbb{E}\left[ X_{2}\right] +\alpha
_{2}^{2}\mathbb{E}\left[ X_{2}^{2}\right]%
\end{array}%
\right) .
\end{eqnarray*}
Define 
\begin{eqnarray*}
\mathbf{v}_{1} &=&\left( 1,\mathbb{E}\left[ X_{2}\right] ,\alpha _{1}+\alpha
_{2}\mathbb{E}\left[ X_{2}\right] \right) ^{\intercal }, \\
\mathbf{v}_{2} &=&\left( \mathbb{E}\left[ X_{2}\right] ,\mathbb{E}\left[
X_{2}^{2}\right] ,\alpha _{1}\mathbb{E}\left[ X_{2}\right] +\alpha _{2}%
\mathbb{E}\left[ X_{2}^{2}\right] \right) ^{\intercal }, \\
\mathbf{v}_{3} &=&\left( \alpha _{1}+\alpha _{2}\mathbb{E}\left[ X_{2}\right]
,\alpha _{1}\mathbb{E}\left[ X_{2}\right] +\alpha _{2}\mathbb{E}\left[
X_{2}^{2}\right] ,\alpha _{1}^{2}+2\alpha _{1}\alpha _{2}\mathbb{E}\left[
X_{2}\right] +\alpha _{2}^{2}\mathbb{E}\left[ X_{2}^{2}\right] \right)
^{\intercal }.
\end{eqnarray*}

$\mathbf{v}_{1},\mathbf{v}_{2}$ and $\mathbf{v}_{3}$ are the first, second
and third column of the matrix $\mathbf{Q}_{XX}.$ We can observe that

\begin{equation*}
\alpha _{1}\mathbf{v}_{1}+\alpha _{2}\mathbf{v}_{2}=\mathbf{v}_{3}.
\end{equation*}

By defintion, $\mathbf{v}_{1},\mathbf{v}_{2}$ and $\mathbf{v}_{3}$ are not
linearly independent. Therefore, $\mathbf{Q}_{XX}$ is not invertible.

\subsection*{(b)}

Let $\mathcal{X}=\left( 1,X_{2}\right) $, and $A=\left[ 
\begin{array}{ccc}
1 & 0 & 0 \\ 
0 & 1 & 0%
\end{array}%
\right] .$As a result, $AX=\mathcal{X}.$Since $\ X_{3}$ is a linear function
of $X_{2}$, the best linear predictor $X^{\intercal }\mathbf{\beta }%
_{1}\equiv \mathcal{X}^{\intercal }\mathbf{\beta }_{2}$ for different $%
\mathbf{\beta }_{1}$ and $\mathbf{\beta }_{2}$. We also have

\begin{eqnarray*}
\mathbf{\beta }_{2} &=&\left( \mathbb{E}\left[ \mathcal{XX}^{\intercal }%
\right] \right) ^{-1}\mathbb{E}\left[ \mathcal{X}Y\right] \\
&=&\left[ 
\begin{array}{cc}
\mathbb{E}\left[ 1\right] & \mathbb{E}\left[ X_{2}\right] \\ 
\mathbb{E}\left[ X_{2}\right] & \mathbb{E}\left[ X_{2}^{2}\right]%
\end{array}%
\right] ^{-1}\left[ 
\begin{array}{c}
\mathbb{E}\left[ Y\right] \\ 
\mathbb{E}\left[ X_{2}Y\right]%
\end{array}%
\right] \\
&=&\frac{1}{\mathbb{E}\left[ X_{2}^{2}\right] -\left( \mathbb{E}\left[ X_{2}%
\right] \right) ^{2}}\left[ 
\begin{array}{cc}
\mathbb{E}\left[ X_{2}^{2}\right] & -\mathbb{E}\left[ X_{2}\right] \\ 
-\mathbb{E}\left[ X_{2}\right] & 1%
\end{array}%
\right] \left[ 
\begin{array}{c}
\mathbb{E}\left[ Y\right] \\ 
\mathbb{E}\left[ X_{2}Y\right]%
\end{array}%
\right] \\
&=&\frac{1}{\mathbb{E}\left[ X_{2}^{2}\right] -\left( \mathbb{E}\left[ X_{2}%
\right] \right) ^{2}}\left[ 
\begin{array}{c}
\mathbb{E}\left[ X_{2}^{2}\right] \mathbb{E}\left[ Y\right] -\mathbb{E}\left[
X_{2}\right] \mathbb{E}\left[ X_{2}Y\right] \\ 
\mathbb{E}\left[ X_{2}Y\right] -\mathbb{E}\left[ X_{2}\right] \mathbb{E}%
\left[ Y\right]%
\end{array}%
\right] \\
&=&\left[ 
\begin{array}{c}
\frac{\mathbb{E}\left[ X_{2}^{2}\right] \mathbb{E}\left[ Y\right] -\mathbb{E}%
\left[ X_{2}\right] \mathbb{E}\left[ X_{2}Y\right] }{\mathbb{E}\left[
X_{2}^{2}\right] -\left( \mathbb{E}\left[ X_{2}\right] \right) ^{2}} \\ 
\frac{\mathbb{E}\left[ X_{2}Y\right] -\mathbb{E}\left[ X_{2}\right] \mathbb{E%
}\left[ Y\right] }{\mathbb{E}\left[ X_{2}^{2}\right] -\left( \mathbb{E}\left[
X_{2}\right] \right) ^{2}}%
\end{array}%
\right]
\end{eqnarray*}

The best linear predictor of $Y$ given $X$ is

\begin{eqnarray*}
X^{\intercal }\mathbf{\beta }_{1} &=&\mathcal{X}^{\intercal }\mathbf{\beta }%
_{2} \\
&=&\left( AX\right) ^{\intercal }\mathbf{\beta }_{2} \\
&=&X^{\intercal }A^{\intercal }\left[ 
\begin{array}{c}
\frac{\mathbb{E}\left[ X_{2}^{2}\right] \mathbb{E}\left[ Y\right] -\mathbb{E}%
\left[ X_{2}\right] \mathbb{E}\left[ X_{2}Y\right] }{\mathbb{E}\left[
X_{2}^{2}\right] -\left( \mathbb{E}\left[ X_{2}\right] \right) ^{2}} \\ 
\frac{\mathbb{E}\left[ X_{2}Y\right] -\mathbb{E}\left[ X_{2}\right] \mathbb{E%
}\left[ Y\right] }{\mathbb{E}\left[ X_{2}^{2}\right] -\left( \mathbb{E}\left[
X_{2}\right] \right) ^{2}}%
\end{array}%
\right] \\
&=&X^{\intercal }\left[ 
\begin{array}{c}
\frac{\mathbb{E}\left[ X_{2}^{2}\right] \mathbb{E}\left[ Y\right] -\mathbb{E}%
\left[ X_{2}\right] \mathbb{E}\left[ X_{2}Y\right] }{\mathbb{E}\left[
X_{2}^{2}\right] -\left( \mathbb{E}\left[ X_{2}\right] \right) ^{2}} \\ 
\frac{\mathbb{E}\left[ X_{2}Y\right] -\mathbb{E}\left[ X_{2}\right] \mathbb{E%
}\left[ Y\right] }{\mathbb{E}\left[ X_{2}^{2}\right] -\left( \mathbb{E}\left[
X_{2}\right] \right) ^{2}} \\ 
0%
\end{array}%
\right]
\end{eqnarray*}

\section*{Excercise 3.3}

\begin{eqnarray*}
X^{\intercal }\mathbf{\hat{e}} &\mathbf{=}&X^{\intercal }\left( Y-X\hat{\beta%
}\right) \\
&=&X^{\intercal }\left( Y-X\left( X^{\intercal }X\right) ^{-1}X^{\intercal
}Y\right) \\
&=&X^{\intercal }Y-\underset{I}{\underbrace{X^{\intercal }X\left(
X^{\intercal }X\right) ^{-1}}}X^{\intercal }Y \\
&=&X^{\intercal }Y-X^{\intercal }Y \\
&=&0
\end{eqnarray*}

\section*{Excercise 3.12}

Only (3.54) and (3.53) can be estimated by OLS. Since $\mathbf{D}_{1}+%
\mathbf{D}_{2}=\mathbf{1}_{n}$, there are perfect collinearity in (3.52),
which violates the assumption of OLS. (3.52) has regressors $\mathbf{D}_{1},%
\mathbf{D}_{2}$ and $\mathbf{1}_{n}$. (3.53)  has regressors $\mathbf{D}_{1}$
and $\mathbf{D}_{2}$ . (3.53) has regressors $\mathbf{D}_{1}$ and $\mathbf{1}%
_{n}$. $\ $

\begin{eqnarray*}
a\mathbf{D}_{1}+b\mathbf{D}_{2} &=&a\mathbf{D}_{1}+b\left( \mathbf{1}_{n}-%
\mathbf{D}_{1}\right)  \\
&=&\left( a-b\right) \mathbf{D}_{1}+b\mathbf{1}_{n}=0\quad iffa=b=0
\end{eqnarray*}

$\mathbf{D}_{1}$ and $\mathbf{D}_{2}$ are linearly independnent, but $%
\mathbf{D}_{1},\mathbf{D}_{2}$ and $\mathbf{1}_{n}$\textbf{\ }are linearly
dependent.

\subsection*{(a)}

No. $\mathbf{D}_{1}$ alone gives the same information as $\mathbf{D}_{1}$
and $\mathbf{D}_{2}.$

For men, $\alpha _{1}=\mu +\phi .$ For women, $\alpha _{2}=\mu .$ So we have

\begin{eqnarray*}
\alpha _{1} &=&\mu +\phi \\
\alpha _{2} &=&\mu
\end{eqnarray*}

or

\begin{eqnarray*}
\mu &=&\alpha _{2} \\
\phi &=&\alpha _{1}-\alpha _{2}
\end{eqnarray*}

\subsection*{(b)}

The number of non-zero elements in $\mathbf{D}_{1}$ is the number of men and
The number of non-zero elements in $\mathbf{D}_{2}$ is the number of women.

\begin{eqnarray*}
\mathbf{1}^{\intercal }\mathbf{D}_{1} &=&n_{1} \\
\mathbf{1}^{\intercal }\mathbf{D}_{2} &=&n_{2}
\end{eqnarray*}

\section*{Excercise 3.19}

\begin{eqnarray*}
\tilde{e}_{i} &=&Y_{i}-\tilde{Y}_{i} \\
&=&\frac{1}{1-h_{ii}}\hat{e}_{i} \\
&=&\frac{1}{1-X_{i}\left( \mathbf{X}^{\intercal }\mathbf{X}\right) ^{-1}X_{i}%
}\hat{e}_{i} \\
&=&\frac{1}{1-\frac{1}{n}}\left( Y_{i}-X_{i}\hat{\beta}\right) \\
&=&\frac{n}{n-1}\left( Y_{i}-\underset{1}{\underbrace{X_{i}}}\left( \underset%
{n}{\underbrace{\mathbf{X}^{\intercal }\mathbf{X}}}\right) ^{-1}\underset{%
\sum Y_{i}}{\underbrace{\mathbf{X}^{\intercal }\mathbf{Y}}}\right) \\
&=&\frac{n}{n-1}\left( Y_{i}-\bar{Y}\right)
\end{eqnarray*}

\end{document}
