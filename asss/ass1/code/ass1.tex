% Options for packages loaded elsewhere
\PassOptionsToPackage{unicode}{hyperref}
\PassOptionsToPackage{hyphens}{url}
%
\documentclass[
]{article}
\usepackage{amsmath,amssymb}
\usepackage{lmodern}
\usepackage{ifxetex,ifluatex}
\ifnum 0\ifxetex 1\fi\ifluatex 1\fi=0 % if pdftex
  \usepackage[T1]{fontenc}
  \usepackage[utf8]{inputenc}
  \usepackage{textcomp} % provide euro and other symbols
\else % if luatex or xetex
  \usepackage{unicode-math}
  \defaultfontfeatures{Scale=MatchLowercase}
  \defaultfontfeatures[\rmfamily]{Ligatures=TeX,Scale=1}
\fi
% Use upquote if available, for straight quotes in verbatim environments
\IfFileExists{upquote.sty}{\usepackage{upquote}}{}
\IfFileExists{microtype.sty}{% use microtype if available
  \usepackage[]{microtype}
  \UseMicrotypeSet[protrusion]{basicmath} % disable protrusion for tt fonts
}{}
\makeatletter
\@ifundefined{KOMAClassName}{% if non-KOMA class
  \IfFileExists{parskip.sty}{%
    \usepackage{parskip}
  }{% else
    \setlength{\parindent}{0pt}
    \setlength{\parskip}{6pt plus 2pt minus 1pt}}
}{% if KOMA class
  \KOMAoptions{parskip=half}}
\makeatother
\usepackage{xcolor}
\IfFileExists{xurl.sty}{\usepackage{xurl}}{} % add URL line breaks if available
\IfFileExists{bookmark.sty}{\usepackage{bookmark}}{\usepackage{hyperref}}
\hypersetup{
  pdftitle={COMM8102 Assignment 1 coding part},
  hidelinks,
  pdfcreator={LaTeX via pandoc}}
\urlstyle{same} % disable monospaced font for URLs
\usepackage[margin=1in]{geometry}
\usepackage{color}
\usepackage{fancyvrb}
\newcommand{\VerbBar}{|}
\newcommand{\VERB}{\Verb[commandchars=\\\{\}]}
\DefineVerbatimEnvironment{Highlighting}{Verbatim}{commandchars=\\\{\}}
% Add ',fontsize=\small' for more characters per line
\usepackage{framed}
\definecolor{shadecolor}{RGB}{248,248,248}
\newenvironment{Shaded}{\begin{snugshade}}{\end{snugshade}}
\newcommand{\AlertTok}[1]{\textcolor[rgb]{0.94,0.16,0.16}{#1}}
\newcommand{\AnnotationTok}[1]{\textcolor[rgb]{0.56,0.35,0.01}{\textbf{\textit{#1}}}}
\newcommand{\AttributeTok}[1]{\textcolor[rgb]{0.77,0.63,0.00}{#1}}
\newcommand{\BaseNTok}[1]{\textcolor[rgb]{0.00,0.00,0.81}{#1}}
\newcommand{\BuiltInTok}[1]{#1}
\newcommand{\CharTok}[1]{\textcolor[rgb]{0.31,0.60,0.02}{#1}}
\newcommand{\CommentTok}[1]{\textcolor[rgb]{0.56,0.35,0.01}{\textit{#1}}}
\newcommand{\CommentVarTok}[1]{\textcolor[rgb]{0.56,0.35,0.01}{\textbf{\textit{#1}}}}
\newcommand{\ConstantTok}[1]{\textcolor[rgb]{0.00,0.00,0.00}{#1}}
\newcommand{\ControlFlowTok}[1]{\textcolor[rgb]{0.13,0.29,0.53}{\textbf{#1}}}
\newcommand{\DataTypeTok}[1]{\textcolor[rgb]{0.13,0.29,0.53}{#1}}
\newcommand{\DecValTok}[1]{\textcolor[rgb]{0.00,0.00,0.81}{#1}}
\newcommand{\DocumentationTok}[1]{\textcolor[rgb]{0.56,0.35,0.01}{\textbf{\textit{#1}}}}
\newcommand{\ErrorTok}[1]{\textcolor[rgb]{0.64,0.00,0.00}{\textbf{#1}}}
\newcommand{\ExtensionTok}[1]{#1}
\newcommand{\FloatTok}[1]{\textcolor[rgb]{0.00,0.00,0.81}{#1}}
\newcommand{\FunctionTok}[1]{\textcolor[rgb]{0.00,0.00,0.00}{#1}}
\newcommand{\ImportTok}[1]{#1}
\newcommand{\InformationTok}[1]{\textcolor[rgb]{0.56,0.35,0.01}{\textbf{\textit{#1}}}}
\newcommand{\KeywordTok}[1]{\textcolor[rgb]{0.13,0.29,0.53}{\textbf{#1}}}
\newcommand{\NormalTok}[1]{#1}
\newcommand{\OperatorTok}[1]{\textcolor[rgb]{0.81,0.36,0.00}{\textbf{#1}}}
\newcommand{\OtherTok}[1]{\textcolor[rgb]{0.56,0.35,0.01}{#1}}
\newcommand{\PreprocessorTok}[1]{\textcolor[rgb]{0.56,0.35,0.01}{\textit{#1}}}
\newcommand{\RegionMarkerTok}[1]{#1}
\newcommand{\SpecialCharTok}[1]{\textcolor[rgb]{0.00,0.00,0.00}{#1}}
\newcommand{\SpecialStringTok}[1]{\textcolor[rgb]{0.31,0.60,0.02}{#1}}
\newcommand{\StringTok}[1]{\textcolor[rgb]{0.31,0.60,0.02}{#1}}
\newcommand{\VariableTok}[1]{\textcolor[rgb]{0.00,0.00,0.00}{#1}}
\newcommand{\VerbatimStringTok}[1]{\textcolor[rgb]{0.31,0.60,0.02}{#1}}
\newcommand{\WarningTok}[1]{\textcolor[rgb]{0.56,0.35,0.01}{\textbf{\textit{#1}}}}
\usepackage{graphicx}
\makeatletter
\def\maxwidth{\ifdim\Gin@nat@width>\linewidth\linewidth\else\Gin@nat@width\fi}
\def\maxheight{\ifdim\Gin@nat@height>\textheight\textheight\else\Gin@nat@height\fi}
\makeatother
% Scale images if necessary, so that they will not overflow the page
% margins by default, and it is still possible to overwrite the defaults
% using explicit options in \includegraphics[width, height, ...]{}
\setkeys{Gin}{width=\maxwidth,height=\maxheight,keepaspectratio}
% Set default figure placement to htbp
\makeatletter
\def\fps@figure{htbp}
\makeatother
\setlength{\emergencystretch}{3em} % prevent overfull lines
\providecommand{\tightlist}{%
  \setlength{\itemsep}{0pt}\setlength{\parskip}{0pt}}
\setcounter{secnumdepth}{-\maxdimen} % remove section numbering
\ifluatex
  \usepackage{selnolig}  % disable illegal ligatures
\fi

\title{COMM8102 Assignment 1 coding part}
\author{\textsc{Yuhao Liu z5097536}}
\date{\vspace{-2.5em}}

\begin{document}
\maketitle

\begin{Shaded}
\begin{Highlighting}[]
\FunctionTok{library}\NormalTok{(readxl)}
\NormalTok{dat }\OtherTok{\textless{}{-}} \FunctionTok{read\_excel}\NormalTok{(}\StringTok{"cps09mar.xlsx"}\NormalTok{)}
\end{Highlighting}
\end{Shaded}

\hypertarget{excercise-3.24}{%
\section{Excercise 3.24}\label{excercise-3.24}}

First we extract the entries from the dataset and construct the
variables referring to section 3.22 and section 3.25 from the textbook.
We only looking at those entries that are single Asian man with less
that 45 years of experience.

\begin{Shaded}
\begin{Highlighting}[]
\CommentTok{\#3.24 }
\CommentTok{\#single Asian man }
\NormalTok{sam }\OtherTok{\textless{}{-}}\NormalTok{ (dat[,}\DecValTok{11}\NormalTok{]}\SpecialCharTok{==}\DecValTok{4}\NormalTok{)}\SpecialCharTok{\&}\NormalTok{(dat[,}\DecValTok{12}\NormalTok{]}\SpecialCharTok{==}\DecValTok{7}\NormalTok{)}\SpecialCharTok{\&}\NormalTok{(dat[,}\DecValTok{2}\NormalTok{]}\SpecialCharTok{==}\DecValTok{0}\NormalTok{)}
\NormalTok{datsam }\OtherTok{\textless{}{-}}\NormalTok{ dat[sam,]}
\CommentTok{\#experience and experience squared}
\NormalTok{experience}\OtherTok{\textless{}{-}}\NormalTok{datsam[,}\DecValTok{1}\NormalTok{]}\SpecialCharTok{{-}}\NormalTok{datsam[,}\DecValTok{4}\NormalTok{]}\SpecialCharTok{{-}}\DecValTok{6}
\NormalTok{exp2}\OtherTok{\textless{}{-}}\NormalTok{experience}\SpecialCharTok{\^{}}\DecValTok{2}\SpecialCharTok{/}\DecValTok{100}
\CommentTok{\#less than 45 year experience}
\NormalTok{sam}\OtherTok{\textless{}{-}}\NormalTok{experience}\SpecialCharTok{\textless{}}\DecValTok{45}
\NormalTok{datsam }\OtherTok{\textless{}{-}}\NormalTok{ datsam[sam,]}
\CommentTok{\#logwage}
\NormalTok{Y}\OtherTok{\textless{}{-}}\FunctionTok{as.matrix}\NormalTok{(}\FunctionTok{log}\NormalTok{(datsam[,}\DecValTok{5}\NormalTok{]}\SpecialCharTok{/}\NormalTok{(datsam[,}\DecValTok{6}\NormalTok{]}\SpecialCharTok{*}\NormalTok{datsam[,}\DecValTok{7}\NormalTok{])))}
\CommentTok{\#new experience and its squared}
\NormalTok{experience}\OtherTok{\textless{}{-}}\NormalTok{experience[sam]}
\NormalTok{exp2}\OtherTok{\textless{}{-}}\NormalTok{experience}\SpecialCharTok{\^{}}\DecValTok{2}\SpecialCharTok{/}\DecValTok{100}
\CommentTok{\#X }
\NormalTok{X}\OtherTok{\textless{}{-}}\FunctionTok{as.matrix}\NormalTok{(}\FunctionTok{cbind}\NormalTok{(datsam[,}\DecValTok{4}\NormalTok{],experience,exp2,}\FunctionTok{matrix}\NormalTok{(}\DecValTok{1}\NormalTok{,}\FunctionTok{nrow}\NormalTok{(datsam),}\DecValTok{1}\NormalTok{)))}
\end{Highlighting}
\end{Shaded}

\hypertarget{a}{%
\subsection{(a)}\label{a}}

Now we estimate 3.49.

\begin{Shaded}
\begin{Highlighting}[]
\CommentTok{\#estimation of 3.49}
\NormalTok{beta349}\OtherTok{\textless{}{-}}\FunctionTok{solve}\NormalTok{(}\FunctionTok{t}\NormalTok{(X)}\SpecialCharTok{\%*\%}\NormalTok{X,}\FunctionTok{t}\NormalTok{(X)}\SpecialCharTok{\%*\%}\NormalTok{Y)}
\FunctionTok{rownames}\NormalTok{(beta349)}\OtherTok{\textless{}{-}}\FunctionTok{c}\NormalTok{(}\StringTok{\textquotesingle{}education\textquotesingle{}}\NormalTok{,}\StringTok{\textquotesingle{}experience\textquotesingle{}}\NormalTok{,}\StringTok{\textquotesingle{}experience\^{}2/100\textquotesingle{}}\NormalTok{,}\StringTok{\textquotesingle{}intercept\textquotesingle{}}\NormalTok{)}
\FunctionTok{print}\NormalTok{(beta349)}
\end{Highlighting}
\end{Shaded}

\begin{verbatim}
##                     earnings
## education         0.14430729
## experience        0.04263326
## experience^2/100 -0.09505636
## intercept         0.53089068
\end{verbatim}

We can see that the coefficients are same as 3.49. The \(R^2\) and
\textbf{Sum of squared errors} are

\begin{Shaded}
\begin{Highlighting}[]
\CommentTok{\#R\^{}2 and SSE}
\NormalTok{Y\_hat}\OtherTok{\textless{}{-}}\FunctionTok{as.matrix}\NormalTok{(}\FloatTok{0.144}\SpecialCharTok{*}\NormalTok{datsam}\SpecialCharTok{$}\NormalTok{education}\FloatTok{+0.043}\SpecialCharTok{*}\NormalTok{experience}\FloatTok{{-}0.095}\SpecialCharTok{/}\DecValTok{100}\SpecialCharTok{*}\NormalTok{experience}\SpecialCharTok{\^{}}\DecValTok{2}\FloatTok{+0.531}\NormalTok{)}
\CommentTok{\#y bar}
\NormalTok{Y\_bar}\OtherTok{\textless{}{-}}\FunctionTok{mean}\NormalTok{(Y)}
\CommentTok{\#R\^{}2}
\NormalTok{R\_squared}\OtherTok{\textless{}{-}}\FunctionTok{sum}\NormalTok{((Y\_hat}\SpecialCharTok{{-}}\NormalTok{Y\_bar)}\SpecialCharTok{\^{}}\DecValTok{2}\NormalTok{)}\SpecialCharTok{/}\FunctionTok{sum}\NormalTok{((Y}\SpecialCharTok{{-}}\NormalTok{Y\_bar)}\SpecialCharTok{\^{}}\DecValTok{2}\NormalTok{)}
\CommentTok{\#sum of squared errors}
\NormalTok{SSE}\OtherTok{\textless{}{-}}\FunctionTok{sum}\NormalTok{((Y}\SpecialCharTok{{-}}\NormalTok{Y\_hat)}\SpecialCharTok{\^{}}\DecValTok{2}\NormalTok{)}
\CommentTok{\#print result}
\FunctionTok{cat}\NormalTok{(}\StringTok{\textquotesingle{}R\^{}2 = \textquotesingle{}}\NormalTok{, R\_squared)}
\end{Highlighting}
\end{Shaded}

\begin{verbatim}
## R^2 =  0.3886973
\end{verbatim}

\begin{Shaded}
\begin{Highlighting}[]
\FunctionTok{cat}\NormalTok{(}\StringTok{\textquotesingle{}SSE = \textquotesingle{}}\NormalTok{, SSE)}
\end{Highlighting}
\end{Shaded}

\begin{verbatim}
## SSE =  82.50921
\end{verbatim}

The \(R^2\) is \(0.39\) and \(SSE\) is \(82.5\).

\hypertarget{b}{%
\subsection{(b)}\label{b}}

Now, we try the residual regression approach. First, we regress
\textbf{log(wage)} on \textbf{experience and its square}.

\begin{Shaded}
\begin{Highlighting}[]
\CommentTok{\#regress logwage on experience and its sqaure}
\NormalTok{X1}\OtherTok{\textless{}{-}}\FunctionTok{as.matrix}\NormalTok{(}\FunctionTok{cbind}\NormalTok{(experience,exp2,}\FunctionTok{matrix}\NormalTok{(}\DecValTok{1}\NormalTok{,}\FunctionTok{nrow}\NormalTok{(datsam),}\DecValTok{1}\NormalTok{)))}
\NormalTok{beta1}\OtherTok{\textless{}{-}}\FunctionTok{solve}\NormalTok{(}\FunctionTok{t}\NormalTok{(X1)}\SpecialCharTok{\%*\%}\NormalTok{X1,}\FunctionTok{t}\NormalTok{(X1)}\SpecialCharTok{\%*\%}\NormalTok{Y)}
\end{Highlighting}
\end{Shaded}

Then, we we regress \textbf{education} on \textbf{experience and its
square}.

\begin{Shaded}
\begin{Highlighting}[]
\CommentTok{\#regress education on experience and its sqaure}
\NormalTok{beta2}\OtherTok{\textless{}{-}}\FunctionTok{solve}\NormalTok{(}\FunctionTok{t}\NormalTok{(X1)}\SpecialCharTok{\%*\%}\NormalTok{X1,}\FunctionTok{t}\NormalTok{(X1)}\SpecialCharTok{\%*\%}\FunctionTok{as.matrix}\NormalTok{(datsam[,}\DecValTok{4}\NormalTok{]))}
\end{Highlighting}
\end{Shaded}

Finally, we regress the residuals on the residuals.

\begin{Shaded}
\begin{Highlighting}[]
\CommentTok{\#residuals and regression}
\NormalTok{e\_1tilda}\OtherTok{\textless{}{-}}\NormalTok{Y}\SpecialCharTok{{-}}\NormalTok{X1}\SpecialCharTok{\%*\%}\NormalTok{beta1}
\NormalTok{x\_2tilda}\OtherTok{\textless{}{-}}\FunctionTok{as.matrix}\NormalTok{(datsam[,}\DecValTok{4}\NormalTok{]}\SpecialCharTok{{-}}\NormalTok{X1}\SpecialCharTok{\%*\%}\NormalTok{beta2)}
\NormalTok{xres}\OtherTok{\textless{}{-}}\FunctionTok{as.matrix}\NormalTok{(}\FunctionTok{cbind}\NormalTok{(x\_2tilda,}\FunctionTok{matrix}\NormalTok{(}\DecValTok{1}\NormalTok{,}\FunctionTok{nrow}\NormalTok{(datsam),}\DecValTok{1}\NormalTok{)))}
\NormalTok{xxres}\OtherTok{\textless{}{-}}\FunctionTok{t}\NormalTok{(xres)}\SpecialCharTok{\%*\%}\NormalTok{xres}
\NormalTok{xyres}\OtherTok{\textless{}{-}}\FunctionTok{t}\NormalTok{(xres)}\SpecialCharTok{\%*\%}\NormalTok{e\_1tilda}
\NormalTok{beta2\_hat}\OtherTok{\textless{}{-}}\FunctionTok{solve}\NormalTok{(xxres,xyres)}
\FunctionTok{rownames}\NormalTok{(beta2\_hat)}\OtherTok{\textless{}{-}}\FunctionTok{c}\NormalTok{(}\StringTok{\textquotesingle{}beta2\_hat\textquotesingle{}}\NormalTok{,}\StringTok{\textquotesingle{}intercept\textquotesingle{}}\NormalTok{)}
\NormalTok{res\_hat}\OtherTok{\textless{}{-}}\NormalTok{e\_1tilda}\SpecialCharTok{{-}}\NormalTok{xres}\SpecialCharTok{\%*\%}\NormalTok{beta2\_hat}
\FunctionTok{print}\NormalTok{(beta2\_hat)}
\end{Highlighting}
\end{Shaded}

\begin{verbatim}
##                earnings
## beta2_hat  1.443073e-01
## intercept -6.852850e-16
\end{verbatim}

The \(R^2\) and \textbf{Sum of squared errors} in the residual
regression approach is evalued by

\begin{Shaded}
\begin{Highlighting}[]
\CommentTok{\#sse\_NEW}
\NormalTok{SSE\_NEW}\OtherTok{\textless{}{-}}\FunctionTok{sum}\NormalTok{(res\_hat}\SpecialCharTok{\^{}}\DecValTok{2}\NormalTok{)}
\CommentTok{\#new r sqaured}
\NormalTok{rsquared\_new}\OtherTok{\textless{}{-}}\DecValTok{1}\SpecialCharTok{{-}}\NormalTok{SSE\_NEW}\SpecialCharTok{/}\FunctionTok{sum}\NormalTok{((Y}\SpecialCharTok{{-}}\NormalTok{Y\_bar)}\SpecialCharTok{\^{}}\DecValTok{2}\NormalTok{)}
\CommentTok{\#print result}
\FunctionTok{cat}\NormalTok{(}\StringTok{\textquotesingle{}The Re{-}estimate  slope on education is\textquotesingle{}}\NormalTok{, beta2\_hat[}\DecValTok{1}\NormalTok{])}
\end{Highlighting}
\end{Shaded}

\begin{verbatim}
## The Re-estimate  slope on education is 0.1443073
\end{verbatim}

\begin{Shaded}
\begin{Highlighting}[]
\FunctionTok{cat}\NormalTok{(}\StringTok{\textquotesingle{}The new R\^{}2 is\textquotesingle{}}\NormalTok{, rsquared\_new )}
\end{Highlighting}
\end{Shaded}

\begin{verbatim}
## The new R^2 is 0.3893207
\end{verbatim}

\begin{Shaded}
\begin{Highlighting}[]
\FunctionTok{cat}\NormalTok{(}\StringTok{\textquotesingle{}The new SSE is\textquotesingle{}}\NormalTok{, SSE\_NEW )}
\end{Highlighting}
\end{Shaded}

\begin{verbatim}
## The new SSE is 82.505
\end{verbatim}

The slope coefficient on education equals to the value in (3.49). When
we regress \textbf{log(wage)} on \textbf{experience and its square}, the
residuals are the \textbf{log(wage)} change that cannot be explained by
\textbf{experience and its square}. When we regress \textbf{education}
on \textbf{experience and its square}, the residuals are
\textbf{education} change that cannot be explained by \textbf{experience
and its square}. Then we regress the residuals on the residuals, we get
the coefficient that \textbf{log(wage)} change attribute to only
\textbf{education}, which is the same as the explaination of the slope
coefficient of \textbf{education} in OLS.

\hypertarget{c}{%
\subsection{(c)}\label{c}}

The \(R^2\) and \textbf{Sum of squared errors} from part (a) and (b) are
equal. In both regression, the regressors are the same and they are
fully used to explain the total variation, so the residuals should also
be the same.

\hypertarget{excercise-3.26}{%
\section{Excercise 3.26}\label{excercise-3.26}}

\hypertarget{a-1}{%
\subsection{(a)}\label{a-1}}

First, we extract the entries from the dataset.

\begin{Shaded}
\begin{Highlighting}[]
\CommentTok{\#white male hispanic}
\NormalTok{wmh}\OtherTok{\textless{}{-}}\NormalTok{(dat[,}\DecValTok{3}\NormalTok{]}\SpecialCharTok{==}\DecValTok{1}\NormalTok{)}\SpecialCharTok{\&}\NormalTok{(dat[,}\DecValTok{11}\NormalTok{]}\SpecialCharTok{==}\DecValTok{1}\NormalTok{)}\SpecialCharTok{\&}\NormalTok{(dat[,}\DecValTok{2}\NormalTok{]}\SpecialCharTok{==}\DecValTok{0}\NormalTok{)}
\NormalTok{datwmh}\OtherTok{\textless{}{-}}\NormalTok{dat[wmh,]}
\end{Highlighting}
\end{Shaded}

Then construct \textbf{log(wage)}, \textbf{education} and
\textbf{experience and its square} as before.

\begin{Shaded}
\begin{Highlighting}[]
\CommentTok{\#logwage}
\NormalTok{Y}\OtherTok{\textless{}{-}}\FunctionTok{as.matrix}\NormalTok{(}\FunctionTok{log}\NormalTok{(datwmh[,}\DecValTok{5}\NormalTok{]}\SpecialCharTok{/}\NormalTok{(datwmh[,}\DecValTok{6}\NormalTok{]}\SpecialCharTok{*}\NormalTok{datwmh[,}\DecValTok{7}\NormalTok{])))}
\CommentTok{\#education}
\NormalTok{edu}\OtherTok{\textless{}{-}}\FunctionTok{as.matrix}\NormalTok{(datwmh[,}\DecValTok{4}\NormalTok{])}
\CommentTok{\#experience}
\NormalTok{expwmh}\OtherTok{\textless{}{-}}\FunctionTok{as.matrix}\NormalTok{(datwmh[,}\DecValTok{1}\NormalTok{]}\SpecialCharTok{{-}}\NormalTok{datwmh[,}\DecValTok{4}\NormalTok{]}\SpecialCharTok{{-}}\DecValTok{6}\NormalTok{)}
\CommentTok{\#experience sqaure}
\NormalTok{expwmh2}\OtherTok{\textless{}{-}}\FunctionTok{as.matrix}\NormalTok{(expwmh}\SpecialCharTok{\^{}}\DecValTok{2}\SpecialCharTok{/}\DecValTok{100}\NormalTok{)}
\end{Highlighting}
\end{Shaded}

Next, we codify the dummy variable for regions. We have three dummy
variable for the four different regions. The excluded group is when all
dummy variables are \(0\).

\begin{Shaded}
\begin{Highlighting}[]
\CommentTok{\#dummy variables for region}
\CommentTok{\#NE}
\NormalTok{x1}\OtherTok{\textless{}{-}}\FunctionTok{as.numeric}\NormalTok{(datwmh[,}\DecValTok{10}\NormalTok{]}\SpecialCharTok{==}\DecValTok{1}\NormalTok{)}
\CommentTok{\#S}
\NormalTok{x2}\OtherTok{\textless{}{-}}\FunctionTok{as.numeric}\NormalTok{(datwmh[,}\DecValTok{10}\NormalTok{]}\SpecialCharTok{==}\DecValTok{3}\NormalTok{)}
\CommentTok{\#W}
\NormalTok{x3}\OtherTok{\textless{}{-}}\FunctionTok{as.numeric}\NormalTok{(datwmh[,}\DecValTok{10}\NormalTok{]}\SpecialCharTok{==}\DecValTok{4}\NormalTok{)}
\end{Highlighting}
\end{Shaded}

We do the similary thing to the marital status with four dummy
variables.

\begin{Shaded}
\begin{Highlighting}[]
\CommentTok{\#dummy variables for marital}
\CommentTok{\#Married}
\NormalTok{xxx1}\OtherTok{\textless{}{-}}\FunctionTok{as.numeric}\NormalTok{(datwmh[,}\DecValTok{12}\NormalTok{]}\SpecialCharTok{==}\DecValTok{1}\SpecialCharTok{|}\NormalTok{datwmh[,}\DecValTok{12}\NormalTok{]}\SpecialCharTok{==}\DecValTok{2}\SpecialCharTok{|}\NormalTok{datwmh[,}\DecValTok{12}\NormalTok{]}\SpecialCharTok{==}\DecValTok{3}\NormalTok{)}
\CommentTok{\#Widowed}
\NormalTok{xxx2}\OtherTok{\textless{}{-}}\FunctionTok{as.numeric}\NormalTok{(datwmh[,}\DecValTok{12}\NormalTok{]}\SpecialCharTok{==}\DecValTok{4}\NormalTok{)}
\CommentTok{\#Divorced}
\NormalTok{xxx3}\OtherTok{\textless{}{-}}\FunctionTok{as.numeric}\NormalTok{(datwmh[,}\DecValTok{12}\NormalTok{]}\SpecialCharTok{==}\DecValTok{5}\NormalTok{)}
\CommentTok{\#Separated}
\NormalTok{xxx4}\OtherTok{\textless{}{-}}\FunctionTok{as.numeric}\NormalTok{(datwmh[,}\DecValTok{12}\NormalTok{]}\SpecialCharTok{==}\DecValTok{6}\NormalTok{)}
\end{Highlighting}
\end{Shaded}

Finally, we do the regression.

\begin{Shaded}
\begin{Highlighting}[]
\CommentTok{\#regression}
\NormalTok{X}\OtherTok{\textless{}{-}}\FunctionTok{cbind}\NormalTok{(edu,expwmh,expwmh2,x1,x2,x3,xxx1,xxx2,xxx3,xxx4,}\FunctionTok{matrix}\NormalTok{(}\DecValTok{1}\NormalTok{,}\FunctionTok{nrow}\NormalTok{(datwmh),}\DecValTok{1}\NormalTok{))}
\NormalTok{beta326}\OtherTok{\textless{}{-}}\FunctionTok{solve}\NormalTok{(}\FunctionTok{t}\NormalTok{(X)}\SpecialCharTok{\%*\%}\NormalTok{X,}\FunctionTok{t}\NormalTok{(X)}\SpecialCharTok{\%*\%}\NormalTok{Y)}
\FunctionTok{rownames}\NormalTok{(beta326)}\OtherTok{\textless{}{-}}\FunctionTok{c}\NormalTok{(}\StringTok{\textquotesingle{}education\textquotesingle{}}\NormalTok{,}\StringTok{\textquotesingle{}experience\textquotesingle{}}\NormalTok{,}\StringTok{\textquotesingle{}(experience\^{}2)/100\textquotesingle{}}\NormalTok{,}\StringTok{\textquotesingle{}Northeast\textquotesingle{}}\NormalTok{,}
\StringTok{\textquotesingle{}South\textquotesingle{}}\NormalTok{,}\StringTok{\textquotesingle{}West\textquotesingle{}}\NormalTok{,}\StringTok{\textquotesingle{}married\textquotesingle{}}\NormalTok{,}\StringTok{\textquotesingle{}widowed\textquotesingle{}}\NormalTok{,}\StringTok{\textquotesingle{}divorced\textquotesingle{}}\NormalTok{,}\StringTok{\textquotesingle{}separated\textquotesingle{}}\NormalTok{,}\StringTok{\textquotesingle{}intercept\textquotesingle{}}\NormalTok{)}
\FunctionTok{print}\NormalTok{(beta326)}
\end{Highlighting}
\end{Shaded}

\begin{verbatim}
##                       earnings
## education           0.08833357
## experience          0.02792293
## (experience^2)/100 -0.03646362
## Northeast           0.06163684
## South              -0.06753707
## West                0.02011173
## married             0.17796669
## widowed             0.24297545
## divorced            0.07870880
## separated           0.01694743
## intercept           1.19175223
\end{verbatim}

\end{document}
